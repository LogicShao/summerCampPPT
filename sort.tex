\documentclass{ldr-simple-gray}

\usepackage{verbatim}
\usepackage{algorithm}
\usepackage{algorithmic}
\usepackage{hyperref}
\usepackage[utf8]{inputenc}
\usepackage{biblatex}

\title{排序}
\subtitle{快速排序\&归并排序\&堆排序etc.}

\author{邵逸帆$q\omega q$}
\institute[] {
  23电信基地班\\
  兰州大学算法与程序设计集训队
}
\date{\today}
% 标题页图片 插入两张并列图片
\titlegraphic{\includegraphics[height=1.5cm]{./figures/lzu_logo.png} \includegraphics[height=1.5cm]{./images/LZUPAT.png}}

\begin{document}
  \frame{\titlepage}

  \begin{frame}{Intro}
    \begin{itemize}
      \item 排序是将一组数据按照某种顺序重新排列的过程。
      \item 稳定性: 若两个相等的元素在排序前后的相对位置不变,则称排序算法是稳定的。
      \item 时间复杂度:简单计算复杂度一般是通过统计“简单操作”的次数来实现的。基于比较的排序算法的时间复杂度的下界是$O(n\log n)$。
      \item 空间复杂度:排序算法的空间复杂度是指除了输入数据外,算法运行时所需的额外空间。
    \end{itemize}
  \end{frame}

  \begin{frame}{Review}
    通常排序算法可以分为三类:
    \begin{enumerate}
      \item 冒泡排序、选择排序、插入排序
      \item 快速排序、归并排序、堆排序
      \item 计数排序、桶排序、基数排序
    \end{enumerate}

    其中快速排序由于其高效性被广泛使用,而归并排序由于其稳定性被用于外部排序。
  \end{frame}

  \begin{frame}{Comparison}
    % 插入表格
    \begin{table}
      \centering
      \begin{tabular}{|c|c|c|c|c|}
        \hline
        八大排序 & 时间复杂度 & 空间复杂度 & 稳定性 \\
        \hline
        冒泡排序 & $O(n^2)$ & $O(1)$ & 稳定 \\
        选择排序 & $O(n^2)$ & $O(1)$ & 不稳定 \\
        插入排序 & $O(n^2)$ & $O(1)$ & 稳定 \\
        希尔排序 & $O(n^{\frac{3}{2}})$ & $O(1)$ & 不稳定 \\
        归并排序 & $O(n\log n)$ & $O(n)$ & 稳定 \\
        快速排序 & $O(n\log n)$ & $O(1)$ & 不稳定 \\
        堆排序 & $O(n\log n)$ & $O(1)$ & 不稳定 \\
        计数排序 & $O(n+k)$ & $O(k)$ & 稳定 \\
        \hline
      \end{tabular}
    \end{table}
  \end{frame}

  \begin{frame}{Divide and Conquer}
      分治即“分而治之”,就是把一个复杂的问题分成两个或更多的相同或相似的子问题,直到最后子问题可以简单的直接求解,原问题的解即子问题的解的合并。
      \\

      分治主要包含三个步骤:
      \begin{itemize}
        \item 分解:将原问题分解成一系列子问题。
        \item 解决:递归地解决这些子问题。
        \item 合并:将子问题的解合并成原问题的解。
      \end{itemize}

      简单来说就是递归前要做什么(分解),递归后要做什么(合并)。
      \\

      归并排序和快速排序都是基于分治思想的排序算法。
    \end{frame}

  \begin{frame}{Quick Sort}
    \begin{block}{快速排序}
      选择一个基准元素pivot,将数组分成两部分,左边的元素都小于等于pivot,右边的元素都大于等于pivot。递归地对左右两部分做快速排序。
    \end{block}
    \begin{block}{划分}
      \begin{itemize}
        \item 选择一个基准元素pivot。
        \item 用两个指针$p1,p2$分别指向数组的起始位置和结束位置。
        \item 从$p1$开始向右找到第一个大于等于pivot的元素,从$p2$开始向左找到第一个小于等于pivot的元素,交换这两个元素。
        \item 重复上述过程直到$p1$和$p2$相遇。
        \item 将pivot与$p1$指向的元素交换。
      \end{itemize}
    \end{block}
  \end{frame}

  \begin{frame}[fragile]{Quick Sort}
    \begin{verbatim}
void quick_sort(int l, int r) {
  if (l >= r) return;
  int i = l - 1, j = r + 1, x = a[rand() % (r - l + 1) + l];
  while (i < j) {
    do i++; while (a[i] < x);
    do j--; while (a[j] > x);
    if (i < j) std::swap(a[i], a[j]);
  }
  quick_sort(l, j), quick_sort(j + 1, r);
}
    \end{verbatim}
  \end{frame}

  \begin{frame}{第K大数-快速选择算法}
    利用快速排序的思想,我们可以在$O(n)$的时间复杂度内找到数组中的第$k$大数(或者第$k$小数)。
    \begin{itemize}
      \item 当我们划分完成的时候,基准的位置已经确定,假设其位置为i,那么基准就是第i大的数(假设我们是降序排序的)
      \item 若$i==k$,基准即为答案
      \item 若$i<k$,容易看出答案在划分的右区间
      \item 若$i>k$,则答案在左区间
    \end{itemize}
    在最终求得答案时,我们并没有对所有的数组进行排序,而是利用了每一次划分的信息。
  \end{frame}

  \begin{frame}{Merge Sort}
    \begin{block}{归并排序}
      \begin{itemize}
        \item 分解:将数组分成两半。
        \item 解决:递归地对两半进行归并排序。
        \item 合并:将两个有序数组合并成一个有序数组。
      \end{itemize}
    \end{block}

    \begin{block}{合并有序数组}
      \begin{itemize}
        \item 申请一个临时数组tmp,大小为$n$。
        \item 用两个指针$p1,p2$分别指向两个有序数组的起始位置。
        \item 比较$p1,p2$指向的元素,将较小的元素放入tmp中。
        \item 重复上述过程直到两个数组中的元素全部放入tmp中。
        \item 将tmp中的元素复制回原数组。
      \end{itemize}
    \end{block}
  \end{frame}

  \begin{frame}[fragile]{逆序对}
    满足$i<j$且$a[i]>a[j]$的数对称为逆序对。在每次合并的时候,我们考虑左区间元素大于右区间元素的情况,此时左区间的元素个数即为逆序对的个数。
    \begin{verbatim}
void merge(int l, int r) {
  /*...*/ int i = l, j = mid + 1;
  for (int k = l; k <= r; k++) {
    if (j > r || (i <= mid && a[i] <= a[j])) {
      tmp[k] = a[i++];
    } else {
      // 此时a[i~mid]都大于a[j]
      tmp[k] = a[j++], ans += mid - i + 1;
    }
  } /*...*/
}\end{verbatim}
  \end{frame}

  \begin{frame}[fragile]{Merge Sort}
    \begin{verbatim}
void merge_sort(int l, int r) {
  if (l >= r) return;
  int mid = (l + r) / 2;
  merge_sort(l, mid), merge_sort(mid + 1, r);
  int i = l, j = mid + 1, k = l;
  while (i <= mid && j <= r) {
    if (a[i] <= a[j]) tmp[k++] = a[i++];
    else tmp[k++] = a[j++];
  }
  while (i <= mid) tmp[k++] = a[i++];
  while (j <= r) tmp[k++] = a[j++];
  for (int i = l; i <= r; i++) a[i] = tmp[i];
}
    \end{verbatim}
  \end{frame}

  \begin{frame}{Heap}
    我们可以使用堆来优化选择排序,这样选出一个最值的时间复杂度可以降低到$O(\log n)$。
    \begin{itemize}
      \item 堆是一种数据结构。对于任意一个节点,其父节点的值大于等于(或小于等于)其子节点的值。
      \item 堆可以用数组来表示,对于节点$i$,其左儿子为$2i$,右儿子为$2i+1$,父节点为$i/2$。
      \item 堆分为大顶堆和小顶堆。
      \item 堆是一棵完全二叉树,每一个节点的子树都是一个堆。
    \end{itemize}
  \end{frame}

  \begin{frame}{Heap Sort}
    \begin{block}{堆排序}
      \begin{itemize}
        \item 建堆:将数组构建成一个大顶堆。可以证明其为$O(n)$。
        \item 排序:将堆顶元素与最后一个元素交换,然后调整堆。
      \end{itemize}
    \end{block}
    \begin{block}{调整堆}
      \begin{itemize}
        \item 从根节点开始比较左右子节点的值,将较大的子节点与根节点交换。
        \item 递归地对交换后的子节点进行调整。
      \end{itemize}
    \end{block}
    堆排序本质上是一种选择排序。从调整堆的操作中可以看出,堆排序涉及到较远项的交换,从而是不稳定的。
  \end{frame}

  \begin{frame}{伪代码}
    \begin{algorithm}[H]
      \caption{Heap Sort}
      \begin{algorithmic}[1]
        \STATE{BuildHeap()}
        \FOR{$i = n$ to $2$}
          \STATE{Swap(a[1], a[i])}
          \STATE{AdjustHeap(1, i - 1)}
        \ENDFOR
      \end{algorithmic}
    \end{algorithm}

    \begin{algorithm}[H]
      \caption{Build Heap}
      \begin{algorithmic}[1]
        % 添加伪代码的注释
        \FOR[where $i$ is non-leaf node]{$i = n / 2$ to $1$}
          \STATE{AdjustHeap(i, n)}\COMMENT{调整i为根的堆}
        \ENDFOR
      \end{algorithmic}
    \end{algorithm}
  \end{frame}

  \begin{frame}{伪代码}
    \begin{algorithm}[H]
      \caption{Adjust Heap}
      \begin{algorithmic}[1]
        \STATE{$t = a[i]$}\COMMENT{下沉i节点}
        \FOR[从左儿子到最后一个节点]{$j = 2 \times i$ to $n$}
          \IF[选择左右儿子中的较大的]{$j < n$ and $a[j] < a[j + 1]$}
            \STATE{$j++$}
          \ENDIF
          \IF[满足堆的性质就退出]{$t \geq a[j]$}
            \STATE{break}
          \ENDIF
          \STATE{$a[i] = a[j]$}\COMMENT{上浮j节点}
          \STATE{$i = j$}
        \ENDFOR
        \STATE{$a[i] = t$}
      \end{algorithmic}
    \end{algorithm}
  \end{frame}

  \begin{frame}{std::sort in STL}
    \begin{block}{void sort(RandomIt first, RandomIt last, Compare comp);}
      这是std::sort的函数原型。其中RandomIt是一个随机访问迭代器,排序区间是左闭右开的;Compare是一个可调用对象,用于比较待排序的元素,默认为std::less<T>()。
    \end{block}

    \begin{block}{STL中的std::sort函数是非常高效的}
      \begin{itemize}
        \item C++ STL中的sort函数是基于快速排序的。
        \item 对于小规模数据,sort函数会使用插入排序。
        \item 对于大规模数据,sort函数会使用快速排序。
        \item 对于近乎有序的数据,sort函数会使用三路快速排序。
      \end{itemize}
    \end{block}
  \end{frame}

  \begin{frame}[fragile]{Example}
    \begin{block}{例如}
      \begin{verbatim}
int a[1000000], n; // 对于数组a进行排序
std::sort(a, a + n); // 默认升序排序
std::sort(a, a + n, std::greater<int>()); // 降序排序

std::vector<int> v; // 对于vector v进行排序
std::sort(v.begin(), v.end());\end{verbatim}
    \end{block}
    对于std::pair和std::tuple这些已经实现比较的类型也可以直接排序。
    \newline

    当然,STL还提供了诸如 std::stable\_sort, std::partial\_sort, std::nth\_element, std::is\_sorted等函数。
  \end{frame}

  \begin{frame}[fragile]{自定义类型排序}
    对于自定义类型,我们可以重载其比较运算符,或者使用lambda表达式来进行排序。
    \begin{block}{重载operator<}
      \begin{verbatim}
bool operator<(const Node &rhs) const {
  return x < rhs.x || (x == rhs.x && y < rhs.y);
}\end{verbatim}
    \end{block}
    \begin{block}{lambda表达式}
      \begin{verbatim}
std::sort(v.begin(), v.end(),
  [](const Node &a, const Node &b) {
  return a.x < b.x || (a.x == b.x && a.y < b.y);
});\end{verbatim}
    \end{block}
    
    当然一些可调用对象也是可以的,通常它们被称为仿函数(functor)。
  \end{frame}

  \begin{frame}{严格弱序}
    在这里我们需要注意的是:我们自定义的比较函数需要满足``严格弱序",这是 \href{https://zh.cppreference.com/w/cpp/named_req/Compare}{STL} 所要求的。\newline

    严格弱序的定义如下:
    \begin{itemize}
      \item 反自反性:$!comp(a, a)$
      \item 反对称性:$comp(a, b) \Rightarrow !comp(b, a)$
      \item 传递性:$comp(a, b) \&\& comp(b, c) \Rightarrow comp(a, c)$
    \end{itemize}
    定义equiv(a, b)为!comp(a, b) \&\& !comp(b, a),则要满足自反性、对称性、传递性。\newline

    假如我们定义<为<=,那么我们就不满足严格弱序的定义,使用std::sort会出现未定义行为,可能会Runtime Error。
  \end{frame}

  \begin{frame}[fragile]{去重}
    对于一个有序的数组,我们很容易对其进行去重。\\当然STL也提供了对应的函数。

    \begin{block}{去重}
      \begin{verbatim}
std::vector<int> v;
std::sort(v.begin(), v.end());
v.erase(std::unique(v.begin(), v.end()), v.end());\end{verbatim}
        这样就完成了对于数组v的去重操作。
    \end{block}
  \end{frame}

  \begin{frame}{离散化}
    离散化是一种将数据映射到连续的整数区间的方法。在一些问题中,我们需要将一些离散的数据映射到连续的整数区间,以便于我们进行操作。
    \begin{itemize}
      \item 离散化的过程是将所有的数据放入一个数组中,然后对数组进行排序。
      \item 然后对于每一个数据,我们可以通过二分查找找到其在排序后的数组中的位置。
      \item 通过这种方法,我们可以将数据映射到$[1,n]$的整数区间。
    \end{itemize}
  \end{frame}

  \begin{frame}[fragile]{离散化}
    \begin{verbatim}
void discrete() {
  std::sort(a + 1, a + n + 1);
  m = std::unique(a + 1, a + n + 1) - a - 1;
}
int query(int x) {
  return std::lower_bound(a + 1, a + m + 1, x) - a;
}
    \end{verbatim}
  \end{frame}

  \begin{frame} % 结束页
    \frametitle{End}
    \begin{center}
      \Huge{$THX\ 4\ Listening!$}
      \emph{:)}
    \end{center}
  \end{frame}
\end{document}