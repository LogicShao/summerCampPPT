\documentclass{ldr-simple-gray}

\usepackage{verbatim}
\usepackage{algorithm}
\usepackage{algorithmic}
\usepackage[utf8]{inputenc}
\usepackage{biblatex}

\title{排序}
\subtitle{快速排序\&归并排序\&堆排序etc.}

\author{邵逸帆$q\omega q$}
\institute[] {
  23电信基地班\\
  兰州大学算法与程序设计集训队
}
\date{\today}
% 标题页图片 插入两张并列图片
\titlegraphic{\includegraphics[height=1.5cm]{./figures/lzu_logo.png} \includegraphics[height=1.5cm]{./images/LZUPAT.png}}

\begin{document}
  \frame{\titlepage}

  \begin{frame}{Intro}
    \begin{itemize}
      \item 排序是将一组数据按照某种顺序重新排列的过程。
      \item 稳定性: 若两个相等的元素在排序前后的相对位置不变,则称排序算法是稳定的。
      \item 时间复杂度:简单计算复杂度一般是通过统计“简单操作”的次数来实现的。基于比较的排序算法的时间复杂度的下界是$O(n\log n)$。
      \item 空间复杂度:排序算法的空间复杂度是指除了输入数据外,算法运行时所需的额外空间。
    \end{itemize}
  \end{frame}

  \begin{frame}{Review}
    % 插入表格
    \begin{table}
      \centering
      \begin{tabular}{|c|c|c|c|c|}
        \hline
        八大排序 & 时间复杂度 & 空间复杂度 & 稳定性 \\
        \hline
        冒泡排序 & $O(n^2)$ & $O(1)$ & 稳定 \\
        选择排序 & $O(n^2)$ & $O(1)$ & 不稳定 \\
        插入排序 & $O(n^2)$ & $O(1)$ & 稳定 \\
        希尔排序 & $O(n^{\frac{3}{2}})$ & $O(1)$ & 不稳定 \\
        归并排序 & $O(n\log n)$ & $O(n)$ & 稳定 \\
        快速排序 & $O(n\log n)$ & $O(1)$ & 不稳定 \\
        堆排序 & $O(n\log n)$ & $O(1)$ & 不稳定 \\
        计数排序 & $O(n+k)$ & $O(k)$ & 稳定 \\
        \hline
      \end{tabular}
    \end{table}
  \end{frame}

  \begin{frame}{Divide and Conquer}
      分治即“分而治之”,就是把一个复杂的问题分成两个或更多的相同或相似的子问题,直到最后子问题可以简单的直接求解,原问题的解即子问题的解的合并。\\
      分治主要包含三个步骤:
      \begin{itemize}
        \item 分解:将原问题分解成一系列子问题。
        \item 解决:递归地解决这些子问题。
        \item 合并:将子问题的解合并成原问题的解。
      \end{itemize}
      简单来说就是递归前要做什么(分解),递归后要做什么(合并)。\\
      归并排序和快速排序都是基于分治思想的排序算法。
    \end{frame}

  \begin{frame}{Quick Sort}
    \begin{block}{快速排序}
      选择一个基准元素pivot,将数组分成两部分,左边的元素都小于等于pivot,右边的元素都大于等于pivot。递归地对左右两部分做快速排序。
    \end{block}
    \begin{block}{划分}
      \begin{itemize}
        \item 选择一个基准元素pivot。
        \item 用两个指针$p1,p2$分别指向数组的起始位置和结束位置。
        \item 从$p1$开始向右找到第一个大于等于pivot的元素,从$p2$开始向左找到第一个小于等于pivot的元素,交换这两个元素。
        \item 重复上述过程直到$p1$和$p2$相遇。
        \item 将pivot与$p1$指向的元素交换。
      \end{itemize}
    \end{block}
  \end{frame}

  \begin{frame}[fragile]{Quick Sort}
    \begin{verbatim}
void quick_sort(int l, int r) {
  if (l >= r) return;
  int i = l, j = r, x = a[rand() % (r - l + 1) + l];
  while (i <= j) {
    do i++; while (a[i] < x);
    do j--; while (a[j] > x);
    if (i < j) swap(a[i], a[j]);
  }
  quick_sort(l, j), quick_sort(i, r);
}
    \end{verbatim}
  \end{frame}

  \begin{frame}{Merge Sort}
    \begin{block}{归并排序}
      \begin{itemize}
        \item 分解:将数组分成两半。
        \item 解决:递归地对两半进行归并排序。
        \item 合并:将两个有序数组合并成一个有序数组。
      \end{itemize}
    \end{block}

    \begin{block}{合并有序数组}
      \begin{itemize}
        \item 申请一个临时数组tmp,大小为$n$。
        \item 用两个指针$p1,p2$分别指向两个有序数组的起始位置。
        \item 比较$p1,p2$指向的元素,将较小的元素放入tmp中。
        \item 重复上述过程直到两个数组中的元素全部放入tmp中。
        \item 将tmp中的元素复制回原数组。
      \end{itemize}
    \end{block}
  \end{frame}

  \begin{frame}[fragile]{Merge Sort}
    \begin{verbatim}
void merge_sort(int l, int r) {
  if (l >= r) return;
  int mid = (l + r) >> 1;
  merge_sort(l, mid), merge_sort(mid + 1, r);
  int i = l, j = mid + 1, k = l;
  while (i <= mid && j <= r) {
    if (a[i] <= a[j]) tmp[k++] = a[i++];
    else tmp[k++] = a[j++];
  }
  while (i <= mid) tmp[k++] = a[i++];
  while (j <= r) tmp[k++] = a[j++];
  for (int i = l; i <= r; i++) a[i] = tmp[i];
}
    \end{verbatim}
  \end{frame}

  \begin{frame}{Heap Sort}
    \begin{block}{堆排序}
      \begin{itemize}
        \item 建堆:将数组构建成一个大顶堆。
        \item 排序:将堆顶元素与最后一个元素交换,然后调整堆。
      \end{itemize}
    \end{block}
    \begin{block}{调整堆}
      \begin{itemize}
        \item 从根节点开始比较左右子节点的值,将较大的子节点与根节点交换。
        \item 递归地对交换后的子节点进行调整。
      \end{itemize}
    \end{block}
    堆排序本质上是一种选择排序。
  \end{frame}

  \begin{frame}{伪代码}
    \begin{algorithm}[H]
      \caption{Heap Sort}
      \begin{algorithmic}[1]
        \STATE{BuildHeap()}
        \FOR{$i = n$ to $2$}
          \STATE{Swap(a[1], a[i])}
          \STATE{AdjustHeap(1, i - 1)}
        \ENDFOR
      \end{algorithmic}
    \end{algorithm}

    \begin{algorithm}[H]
      \caption{Build Heap}
      \begin{algorithmic}[1]
        \FOR{$i = n / 2$ to $1$}
          \STATE{AdjustHeap(i, n)}
        \ENDFOR
      \end{algorithmic}
    \end{algorithm}
  \end{frame}

  \begin{frame}{伪代码}
    \begin{algorithm}[H]
      \caption{Adjust Heap}
      \begin{algorithmic}[1]
        \STATE{$t = a[i]$}
        \FOR{$j = 2 \times i$ to $n$}
          \IF{$j < n$ and $a[j] < a[j + 1]$}
            \STATE{$j++$}
          \ENDIF
          \IF{$t \geq a[j]$}
            \STATE{break}
          \ENDIF
          \STATE{$a[i] = a[j]$}
          \STATE{$i = j$}
        \ENDFOR
        \STATE{$a[i] = t$}
      \end{algorithmic}
    \end{algorithm}
  \end{frame}

  \begin{frame} % 结束页
    \frametitle{End}
    \begin{center}
      \Huge{$THX\ 4\ Listening!$}
      \emph{:)}
    \end{center}
  \end{frame}
\end{document}