\documentclass{ldr-simple-gray}

\usepackage{verbatim}

\title{字符串哈希\&字典树}
\subtitle{QwQ}

\author{邵逸帆}
\institute[] {
  23电信基地班\\
  兰州大学算法与程序设计集训队
}
\date{\today}
% 标题页图片 插入两张并列图片
\titlegraphic{\includegraphics[height=1.5cm]{./figures/lzu_logo.png} \includegraphics[height=1.5cm]{./images/LZUPAT.png}}

\begin{document}
  \frame{\titlepage} % 首页

  \section{字符串基础}
  \begin{frame}{char in C/C++}
    \begin{block}{字符类型}
      \begin{itemize}
        \item char类型
        \item char[]类型(char数组)
        \item char*类型(char指针)
      \end{itemize}
    \end{block}
    
    \begin{block}{字符串字面量}
      eg. \texttt{"Hello, World!"}
      \begin{itemize}
        \item type: char[14] in C, const char[14] in C++
        \item size: 14
      \end{itemize}

      Warning: 不建议使用\texttt{char* str = "Hello, World!";}这样的代码,因为对于其的修改会导致未定义行为(因为你在修改只读内存)。
    \end{block}

    我们可以使用字符数组来存储字符串,这很直观。
  \end{frame}

  \begin{frame}{std::string in C++}
    \texttt{std::string}与\texttt{std::vector<char>}非常相似,只不过前者提供了更多的字符串操作函数。当然,相较于\texttt{char[]}更有优势。

    \begin{block}{std::string}
      \begin{itemize}
        \item append() \& operator+=: 字符串拼接
        \item find(): 查找字符串中的某个子串
        \item substr(): 返回字符串的子串
        \item operator== \& < \& > etc.: 字符串比较
      \end{itemize}
    \end{block}

    诸如\texttt{std::to\_string(),std::stoi(),isdigit()}等函数有时候会帮助你更优雅地处理字符串。
  \end{frame}

  \begin{frame}{What is String}
    字符串定义于字符集之上。一个建立了全序关系的集合 $\Sigma$就可以作为一个字符集,其中的元素称为字符。字符串是字符的有限序列。
    \newline\newline
    字符串的长度是指字符串中字符的个数。字符串的长度可以为0。
    \newline\newline
    之后的内容涉及到子串、子序列、前缀、后缀等概念,请注意区分。
  \end{frame}

  \section{字符串哈希}
  \begin{frame}{什么是哈希}
    pass
  \end{frame}

  \section{字典树}
  \begin{frame}{什么是字典树}
    pass
  \end{frame}

  \begin{frame} % 结束页
    \frametitle{End}
    \begin{center}
      \Huge{$THX\ 4\ Listening!$}
      \emph{:)}
    \end{center}
  \end{frame}
\end{document}